\documentclass[]{article}
\usepackage{lmodern}
\usepackage{amssymb,amsmath}
\usepackage{ifxetex,ifluatex}
\usepackage{fixltx2e} % provides \textsubscript
\ifnum 0\ifxetex 1\fi\ifluatex 1\fi=0 % if pdftex
  \usepackage[T1]{fontenc}
  \usepackage[utf8]{inputenc}
\else % if luatex or xelatex
  \ifxetex
    \usepackage{mathspec}
  \else
    \usepackage{fontspec}
  \fi
  \defaultfontfeatures{Ligatures=TeX,Scale=MatchLowercase}
\fi
% use upquote if available, for straight quotes in verbatim environments
\IfFileExists{upquote.sty}{\usepackage{upquote}}{}
% use microtype if available
\IfFileExists{microtype.sty}{%
\usepackage{microtype}
\UseMicrotypeSet[protrusion]{basicmath} % disable protrusion for tt fonts
}{}
\usepackage[margin=1in]{geometry}
\usepackage{hyperref}
\hypersetup{unicode=true,
            pdftitle={Assignment 3},
            pdfauthor={Filip Mellgren},
            pdfborder={0 0 0},
            breaklinks=true}
\urlstyle{same}  % don't use monospace font for urls
\usepackage{graphicx,grffile}
\makeatletter
\def\maxwidth{\ifdim\Gin@nat@width>\linewidth\linewidth\else\Gin@nat@width\fi}
\def\maxheight{\ifdim\Gin@nat@height>\textheight\textheight\else\Gin@nat@height\fi}
\makeatother
% Scale images if necessary, so that they will not overflow the page
% margins by default, and it is still possible to overwrite the defaults
% using explicit options in \includegraphics[width, height, ...]{}
\setkeys{Gin}{width=\maxwidth,height=\maxheight,keepaspectratio}
\IfFileExists{parskip.sty}{%
\usepackage{parskip}
}{% else
\setlength{\parindent}{0pt}
\setlength{\parskip}{6pt plus 2pt minus 1pt}
}
\setlength{\emergencystretch}{3em}  % prevent overfull lines
\providecommand{\tightlist}{%
  \setlength{\itemsep}{0pt}\setlength{\parskip}{0pt}}
\setcounter{secnumdepth}{0}
% Redefines (sub)paragraphs to behave more like sections
\ifx\paragraph\undefined\else
\let\oldparagraph\paragraph
\renewcommand{\paragraph}[1]{\oldparagraph{#1}\mbox{}}
\fi
\ifx\subparagraph\undefined\else
\let\oldsubparagraph\subparagraph
\renewcommand{\subparagraph}[1]{\oldsubparagraph{#1}\mbox{}}
\fi

%%% Use protect on footnotes to avoid problems with footnotes in titles
\let\rmarkdownfootnote\footnote%
\def\footnote{\protect\rmarkdownfootnote}

%%% Change title format to be more compact
\usepackage{titling}

% Create subtitle command for use in maketitle
\newcommand{\subtitle}[1]{
  \posttitle{
    \begin{center}\large#1\end{center}
    }
}

\setlength{\droptitle}{-2em}

  \title{Assignment 3}
    \pretitle{\vspace{\droptitle}\centering\huge}
  \posttitle{\par}
    \author{Filip Mellgren}
    \preauthor{\centering\large\emph}
  \postauthor{\par}
      \predate{\centering\large\emph}
  \postdate{\par}
    \date{2019-02-23}


\begin{document}
\maketitle

{
\setcounter{tocdepth}{2}
\tableofcontents
}
\begin{center}\rule{0.5\linewidth}{\linethickness}\end{center}

Hello,this is a comment from Ismael. Once again.

\section{Theoretical exercises}\label{theoretical-exercises}

\subsection{1}\label{section}

\subsubsection{a:}\label{a}

test Show that \(* = Cov(z_t, \varepsilon_{yt}) \neq 0\).

\begin{itemize}
\item
  Recall the formula for covariance:
  \(Cov(z_t, \varepsilon_{yt}) = E(z_t\varepsilon_{yt})-E(z_t)E(\varepsilon_{yt})\).
  Because \(\varepsilon_{yt} \sim WN(0,\sigma_y^2)\), we obtain:
  \(* = E(z_t\varepsilon_{yt})\).
\item
  Next, expand the the expression for \(y_t\) in the expression for
  \(z_t\):
  \(* = E[(-b_{21}[(b_{12}z_t + \gamma_{11}y_{t-1} + \gamma_{12}z_{t-1} + \varepsilon_{yt}] + \gamma_{21}y_{t-1}+\gamma_{22}z_{t-1} + \varepsilon_{zt})\varepsilon_{yt}]\).
\item
  Now distribute \(\varepsilon_{yt}\) over the system:
  \(* = E(-b_{21}[(b_{12}z_t + \gamma_{11}y_{t-1} + \gamma_{12}z_{t-1} + \varepsilon_{yt}]\varepsilon_{yt} + \gamma_{21}y_{t-1}\varepsilon_{yt} + \gamma_{22}z_{t-1}\varepsilon_{yt} + \varepsilon_{zt}\varepsilon_{yt})\)
\item
  Expand the expectation operator to a sum:
  \(* = E(-b_{21}[(b_{12}z_t + \gamma_{11}y_{t-1} + \gamma_{12}z_{t-1} + \varepsilon_{yt}]\varepsilon_{yt}) + E(\gamma_{21}y_{t-1}\varepsilon_{yt}) + E(\gamma_{22}z_{t-1}\varepsilon_{yt}) + E(\varepsilon_{zt}\varepsilon_{yt})\).
\item
  Exploit intertemporal independence and that \(\varepsilon_{yt}\) and
  \(\varepsilon_{zt}\) are independent:
  \(* = E(-b_{21}[(b_{12}z_t + \gamma_{11}y_{t-1} + \gamma_{12}z_{t-1} + \varepsilon_{yt}]\varepsilon_{yt})\)
\item
  Distibute \(\varepsilon_{yt}\):
  \(* = -b_{21}E([(b_{12}z_t\varepsilon_{yt} + \gamma_{11}y_{t-1}\varepsilon_{yt} + \gamma_{12}z_{t-1}\varepsilon_{yt} + \varepsilon_{yt}\varepsilon_{yt}])\)
\item
  Expand the expectation:
  \(*=-b_{21}[E(b_{12}z_t\varepsilon_{yt})+ E(\gamma_{11}y_{t-1}\varepsilon_{yt}) + E(\gamma_{12}z_{t-1}\varepsilon_{yt}) + E(\varepsilon_{yt}^2)]\)
\item
  What remains after exploiting independence is
  \(*=-b_{21} E(\varepsilon_{yt}^2) = -b_{21}\sigma_y^2 \neq 0\) QED.
\end{itemize}

The implications on estimation are that estimates will be inefficient
and baised.

\subsubsection{b}\label{b}

Firstly, we express (1) in the following matrix form:

\[BX_t=\Gamma_1X_{t-1} + \varepsilon_t\] Where \[
   B=
  \left[ {\begin{array}{cc}
   1 & b_{12} \\
   b_{21} & 1 \\
  \end{array} } \right]
\] , \[
   X_t=
  \left[ {\begin{array}{cc}
   y_t  \\
   z_t \\
  \end{array} } \right]
\] \[
   \Gamma_1=
  \left[ {\begin{array}{cc}
   \gamma_{11} & \gamma_{12} \\
   \gamma_{21} & \gamma_{22} \\
  \end{array} } \right]
\] \[
   \varepsilon_t=
  \left[ {\begin{array}{cc}
   \varepsilon_{y,t}  \\
   \varepsilon_{z,t} \\
  \end{array} } \right]
\]

Multiplying both sides by the inverse og \(B\) makes us obtain the VAR
model in standard form: \[x_t = A_1x_{t-1}+e_t\] where
\(A_1 =B^{-1}\Gamma_1\), \(e_t = B^{-1}\varepsilon_t\)

\subsubsection{c}\label{c}

\paragraph{(i)}\label{i}

In this particular case, \[
   B=
  \left[ {\begin{array}{cc}
   1 & b_{12} \\
   0 & 1 \\
  \end{array} } \right]
\]. We also know that \(BA_1 =\Gamma_1\), therefore we can express
\(\Gamma_1\) as:

\[\Gamma_1=
  \left[ {\begin{array}{cc}
   \gamma_{11} & \gamma_{12} \\
   \gamma_{21} & \gamma_{22} \\
  \end{array} } \right]=
    \left[ {\begin{array}{cc}
   0.6-0.1b_{12} & 0.2-0.8b_{12} \\
   -0.1 & -0.8 \\
  \end{array} } \right]
  \]

Also, we know \(e_t=B\varepsilon_t\), then:

\[e_t=
    \left[ {\begin{array}{cc}
   e_{1,t}  \\
   e_{2,t} \\
  \end{array} } \right]=
  \left[ {\begin{array}{cc}
   \varepsilon_{y,t}-\varepsilon_{z,t}b_{12} \\
   -\varepsilon_{z,t} \\
  \end{array} } \right]
  \]

From where we can get the following matrix of covariances expressed as
variances of the structural errors and parameter \(b_{12}\):

\[ \Sigma_e=
    \left[ {\begin{array}{cc}
   \sigma_y^2+(b_{12}^2+b_{12})\sigma_z^2 & b_{12}\sigma_z^2 \\
   b_{12}\sigma_z^2 & \sigma_z^2 \\
  \end{array} } \right]=
  \left[ {\begin{array}{cc}
   1 & 0.5 \\
   0.5 & 2 \\
  \end{array} } \right]
\]

With some algebra we find that \(b_{12}=0.25\), \(\sigma_y^2=3/8\),
\(\sigma_z^2=2\) and \[\Gamma_1=
  \left[ {\begin{array}{cc}
  0.575 & 0 \\
  -0.1 & -0.8 \\
  \end{array} } \right]\]

\paragraph{(ii)}\label{ii}

First, we define \[B=
\left[ {\begin{array}{cc}
1 & b \\
b & 1 \\
\end{array} } \right]\], where \(b=b_{12}=b_{21}\). Additionally, as
explained in p.317 of the book, the covariance matrix of the reduce form
can be expressed as:

\[\Sigma_e=B^{-1}\Sigma_\varepsilon (B^{-1})^T\]

Given \(B\) is a symetric matrix, the following expression holds:

\[B\Sigma_eB=BB^{-1}\Sigma_\varepsilon (B^{-1})^TB=\Sigma_\varepsilon\]
Where the extremes' expressions are equivalent to:

{[} \left[ {\begin{array}{cc}
   2b^2-b+1 & -0.5b^2+3b-0.5 \\
  -0.5b^2+3b-0.5  & b^2-b+2 \\
  \end{array} } \right]=

\begin{verbatim}
  \left[ {\begin{array}{cc}
\end{verbatim}

\sigma\_y\^{}2 \& 0 \textbackslash{} 0 \& \sigma\_z\^{}2
\textbackslash{} \textbackslash{}end\{array\} \} \right{]}= {]}

Solving for \(b\) in \(-0.5b^2+3b-0.5=0\) we get the following two sets
of solutions:

\[\theta_i=(b_1, \sigma_y^2, \sigma_z^2, \gamma_{11}, \gamma_{12}, \gamma_{21}, \gamma_{22})_i=(0.172, 0.887, 1.858, 0.583, 0.063, 0.003, -0.766)\]

\section{\texorpdfstring{\[\theta_{ii}=(b_1, \sigma_y^2, \sigma_z^2, \gamma_{11}, \gamma_{12}, \gamma_{21}, \gamma_{22})_{ii}=(5.828, 63.112, 30.142, 0.0172, -4.463, 3.397, 0.366)\]}{\textbackslash{}theta\_\{ii\}=(b\_1, \textbackslash{}sigma\_y\^{}2, \textbackslash{}sigma\_z\^{}2, \textbackslash{}gamma\_\{11\}, \textbackslash{}gamma\_\{12\}, \textbackslash{}gamma\_\{21\}, \textbackslash{}gamma\_\{22\})\_\{ii\}=(5.828, 63.112, 30.142, 0.0172, -4.463, 3.397, 0.366)}}\label{theta_iib_1-sigma_y2-sigma_z2-gamma_11-gamma_12-gamma_21-gamma_22_ii5.828-63.112-30.142-0.0172--4.463-3.397-0.366}

\subsubsection{d}\label{d}

\includegraphics{Assignment_3_files/figure-latex/unnamed-chunk-1-1.pdf}
\includegraphics{Assignment_3_files/figure-latex/unnamed-chunk-1-2.pdf}
\includegraphics{Assignment_3_files/figure-latex/unnamed-chunk-2-1.pdf}
\includegraphics{Assignment_3_files/figure-latex/unnamed-chunk-2-2.pdf}
There's a unit root if solutions of the system \(det(I - zA_1) = 0\) lie
on the unit circle. A unit root means \(z = (1,1)\), then \[I - A_1 =
\left[ {\begin{array}{cc} 
1- a_{11} & -a_{12} \\ 
-a_{21} & 1-a_{22} \\ 
\end{array} } \right]\]

For simplicity, take \(A_1 = I\), then the determinant:
\[ det(I-A_1) = 0\].

\includegraphics{Assignment_3_files/figure-latex/unnamed-chunk-3-1.pdf}
\includegraphics{Assignment_3_files/figure-latex/unnamed-chunk-3-2.pdf}
The shock will never fade as the impacts seem to not return to zero and
remain constant. \#\# 2

\section{Empirical exercises}\label{empirical-exercises}

Do exercises 10a-10g in the textbook (p.340)

\begin{itemize}
\item
  Remark 1: It is possible that the values you obtain for the
  F-statistics, p-values and correlations are different than those
  reported since the sample is extended. However, the main conclusions
  should be the same.
\item
  Remark 2: Exercise d. is optional and so is the part on the forecast
  error variance in e. (but you could use the command fevd in STATA to
  answer these questions).
\item
  Remark 3: You find the appropriate specifications for the variables st
  , \(\Delta\)lip, and \(\Delta\)ur described in the text to exercise 9
  (p.339).
\end{itemize}

\subsubsection{10:}\label{section-1}

Estimate the three-VAR beginning in 1961Q1 and use the ordering such
that \(\Delta lip_t\) is causally prior to \(\Delta ur_t\) and that
\(\Delta ur_t\) is causally prior to \(s_t\).

We begin by defining the variables we are going to include in our
analysis. We create
\(dlip = log(indprod_t) - log(indprod_{t-1}), dur = urate_t - urate_{t-1}\)
and \(s = r10 - tbill\).

In the context of chapter 5, we assume that staionarity holds.
Additionally, it is provided for us that the appropriate lag length is
3. The result of the var estimation is as follows:

\begin{verbatim}
## 
## VAR Estimation Results:
## ========================= 
## Endogenous variables: s, dur, dlip 
## Deterministic variables: none 
## Sample size: 231 
## Log Likelihood: 612.866 
## Roots of the characteristic polynomial:
## 0.9173 0.7824 0.6453 0.463 0.463 0.4479 0.4479 0.1831 0.1831
## Call:
## VAR(y = ., p = 3, type = "none")
## 
## 
## Estimation results for equation s: 
## ================================== 
## s = s.l1 + dur.l1 + dlip.l1 + s.l2 + dur.l2 + dlip.l2 + s.l3 + dur.l3 + dlip.l3 
## 
##         Estimate Std. Error t value Pr(>|t|)    
## s.l1     1.08636    0.06711  16.188  < 2e-16 ***
## dur.l1   0.58848    0.19168   3.070  0.00241 ** 
## dlip.l1  0.50432    3.78403   0.133  0.89409    
## s.l2    -0.31910    0.09737  -3.277  0.00122 ** 
## dur.l2  -0.24687    0.21092  -1.170  0.24307    
## dlip.l2  2.10670    4.09152   0.515  0.60714    
## s.l3     0.18490    0.06767   2.732  0.00679 ** 
## dur.l3   0.30594    0.18962   1.613  0.10808    
## dlip.l3  0.35224    3.71775   0.095  0.92460    
## ---
## Signif. codes:  0 '***' 0.001 '**' 0.01 '*' 0.05 '.' 0.1 ' ' 1
## 
## 
## Residual standard error: 0.5104 on 222 degrees of freedom
## Multiple R-Squared: 0.9281,  Adjusted R-squared: 0.9252 
## F-statistic: 318.3 on 9 and 222 DF,  p-value: < 2.2e-16 
## 
## 
## Estimation results for equation dur: 
## ==================================== 
## dur = s.l1 + dur.l1 + dlip.l1 + s.l2 + dur.l2 + dlip.l2 + s.l3 + dur.l3 + dlip.l3 
## 
##          Estimate Std. Error t value Pr(>|t|)    
## s.l1     0.006923   0.031567   0.219   0.8266    
## dur.l1   0.522127   0.090161   5.791 2.37e-08 ***
## dlip.l1 -4.056028   1.779886  -2.279   0.0236 *  
## s.l2    -0.009138   0.045798  -0.200   0.8420    
## dur.l2   0.056733   0.099211   0.572   0.5680    
## dlip.l2  2.175020   1.924519   1.130   0.2596    
## s.l3    -0.016687   0.031829  -0.524   0.6006    
## dur.l3   0.028446   0.089192   0.319   0.7501    
## dlip.l3  1.033190   1.748709   0.591   0.5552    
## ---
## Signif. codes:  0 '***' 0.001 '**' 0.01 '*' 0.05 '.' 0.1 ' ' 1
## 
## 
## Residual standard error: 0.2401 on 222 degrees of freedom
## Multiple R-Squared: 0.4751,  Adjusted R-squared: 0.4538 
## F-statistic: 22.32 on 9 and 222 DF,  p-value: < 2.2e-16 
## 
## 
## Estimation results for equation dlip: 
## ===================================== 
## dlip = s.l1 + dur.l1 + dlip.l1 + s.l2 + dur.l2 + dlip.l2 + s.l3 + dur.l3 + dlip.l3 
## 
##           Estimate Std. Error t value Pr(>|t|)    
## s.l1     0.0009883  0.0015925   0.621   0.5355    
## dur.l1  -0.0063116  0.0045486  -1.388   0.1666    
## dlip.l1  0.5404560  0.0897942   6.019 7.19e-09 ***
## s.l2     0.0010944  0.0023105   0.474   0.6362    
## dur.l2   0.0062905  0.0050051   1.257   0.2101    
## dlip.l2 -0.0650182  0.0970909  -0.670   0.5038    
## s.l3    -0.0002954  0.0016058  -0.184   0.8542    
## dur.l3   0.0044926  0.0044997   0.998   0.3192    
## dlip.l3  0.1584778  0.0882214   1.796   0.0738 .  
## ---
## Signif. codes:  0 '***' 0.001 '**' 0.01 '*' 0.05 '.' 0.1 ' ' 1
## 
## 
## Residual standard error: 0.01211 on 222 degrees of freedom
## Multiple R-Squared: 0.4889,  Adjusted R-squared: 0.4682 
## F-statistic:  23.6 on 9 and 222 DF,  p-value: < 2.2e-16 
## 
## 
## 
## Covariance matrix of residuals:
##              s       dur       dlip
## s     0.258211  0.026343 -0.0011285
## dur   0.026343  0.056898 -0.0019789
## dlip -0.001128 -0.001979  0.0001467
## 
## Correlation matrix of residuals:
##            s     dur    dlip
## s     1.0000  0.2173 -0.1833
## dur   0.2173  1.0000 -0.6849
## dlip -0.1833 -0.6849  1.0000
\end{verbatim}

We also check for serial correlation using the adjusted Portmanteau test
with 8 lags:

\begin{verbatim}
## 
##  Portmanteau Test (adjusted)
## 
## data:  Residuals of VAR object var
## Chi-squared = 101.67, df = 45, p-value = 2.864e-06
\end{verbatim}

The null of no serial correlation is rejected. Hence, there is still
serial correlation in the data. Anyway, we specified the model with the
number of lags given in the book, so we proceed without further changes.

\subsubsection{10 a:}\label{a-1}

If you perform a test to determine whether \(s_t\) Granger causes
\(\Delta lip_t\), you should find that the F-statistic is 2.44 with a
prob-value of 0.065. How do you interpret this result?

\begin{tabular}{@{\extracolsep{5pt}}lccc} 
\\[-1.8ex]\hline 
\hline \\[-1.8ex] 
Statistic & \multicolumn{1}{c}{N} & \multicolumn{1}{c}{Mean} & \multicolumn{1}{c}{St. Dev.} \\ 
\hline \\[-1.8ex] 
Res.Df & 2 & 225.500 & 2.121 \\ 
Df & 1 & $-$3.000 &  \\ 
F & 1 & 2.774 &  \\ 
Pr(\textgreater F) & 1 & 0.042 &  \\ 
\hline \\[-1.8ex] 
\end{tabular}

The p-value is borderline significant. Assuming the null that no lag of
\(s\) predicts \(dlip\) does not hold, then the meaning is that there is
a lag of \(s\) that does predict \(dlip\). Hence, \(s\) granger causes
\(dlip\). However, it is not clear cut given the p-value.

\subsubsection{10 b:}\label{b-1}

Verify that \(s_t\) Granger causes \(\Delta unemp_t\). You should find
that the F statistic is 5.93 with a prob value of less than 0.001.

\begin{tabular}{@{\extracolsep{5pt}}lccc} 
\\[-1.8ex]\hline 
\hline \\[-1.8ex] 
Statistic & \multicolumn{1}{c}{N} & \multicolumn{1}{c}{Mean} & \multicolumn{1}{c}{St. Dev.} \\ 
\hline \\[-1.8ex] 
Res.Df & 2 & 225.500 & 2.121 \\ 
Df & 1 & $-$3.000 &  \\ 
F & 1 & 4.450 &  \\ 
Pr(\textgreater F) & 1 & 0.005 &  \\ 
\hline \\[-1.8ex] 
\end{tabular}

\subsubsection{10 c:}\label{c-1}

It turns out that the correlation coefficient between \(e_{1t}\) and
\(e_{2t}\) is -0.72. The correlation between \(e_{1t}\) and \(e_{3t}\)
is -0.11 and between \(e_{2t}\) and \(e_{3t}\) is 0.10. Explain why the
ordering of a Choleski composition is likely to be important for
obtaining the impulse responses.

EXPLAIN THE ORDERING

\subsubsection{10 e:}\label{e}

Now estimate the model using the levels of \(lip_t\) and \(ur_t\). Do
you now find a lag length of 5 appropriate?

\begin{verbatim}
## 
## VAR Estimation Results:
## ========================= 
## Endogenous variables: s, dur, dlip 
## Deterministic variables: none 
## Sample size: 229 
## Log Likelihood: 634.837 
## Roots of the characteristic polynomial:
## 0.9312 0.838 0.7113 0.6921 0.6921 0.6647 0.6647 0.6472 0.6472 0.6283 0.6283 0.597 0.597 0.4849 0.4849
## Call:
## VAR(y = ., p = 5, type = "none")
## 
## 
## Estimation results for equation s: 
## ================================== 
## s = s.l1 + dur.l1 + dlip.l1 + s.l2 + dur.l2 + dlip.l2 + s.l3 + dur.l3 + dlip.l3 + s.l4 + dur.l4 + dlip.l4 + s.l5 + dur.l5 + dlip.l5 
## 
##         Estimate Std. Error t value Pr(>|t|)    
## s.l1     1.11837    0.06839  16.353  < 2e-16 ***
## dur.l1   0.68432    0.19940   3.432 0.000719 ***
## dlip.l1  0.52247    3.83414   0.136 0.891737    
## s.l2    -0.41224    0.10160  -4.057 6.96e-05 ***
## dur.l2  -0.21959    0.21194  -1.036 0.301337    
## dlip.l2  4.31175    4.17523   1.033 0.302911    
## s.l3     0.35337    0.10223   3.457 0.000659 ***
## dur.l3   0.32927    0.21304   1.546 0.123689    
## dlip.l3 -1.77888    4.11297  -0.433 0.665811    
## s.l4    -0.24039    0.10036  -2.395 0.017467 *  
## dur.l4  -0.23391    0.21536  -1.086 0.278633    
## dlip.l4  3.28703    4.13872   0.794 0.427950    
## s.l5     0.14193    0.06857   2.070 0.039678 *  
## dur.l5   0.14418    0.19029   0.758 0.449472    
## dlip.l5 -5.51291    3.83101  -1.439 0.151606    
## ---
## Signif. codes:  0 '***' 0.001 '**' 0.01 '*' 0.05 '.' 0.1 ' ' 1
## 
## 
## Residual standard error: 0.5044 on 214 degrees of freedom
## Multiple R-Squared: 0.9319,  Adjusted R-squared: 0.9272 
## F-statistic: 195.4 on 15 and 214 DF,  p-value: < 2.2e-16 
## 
## 
## Estimation results for equation dur: 
## ==================================== 
## dur = s.l1 + dur.l1 + dlip.l1 + s.l2 + dur.l2 + dlip.l2 + s.l3 + dur.l3 + dlip.l3 + s.l4 + dur.l4 + dlip.l4 + s.l5 + dur.l5 + dlip.l5 
## 
##          Estimate Std. Error t value Pr(>|t|)    
## s.l1     0.023731   0.031893   0.744  0.45765    
## dur.l1   0.451130   0.092993   4.851 2.36e-06 ***
## dlip.l1 -5.096861   1.788062  -2.850  0.00479 ** 
## s.l2    -0.042737   0.047383  -0.902  0.36810    
## dur.l2   0.049793   0.098839   0.504  0.61494    
## dlip.l2  2.100112   1.947130   1.079  0.28199    
## s.l3     0.029039   0.047673   0.609  0.54309    
## dur.l3   0.110662   0.099352   1.114  0.26660    
## dlip.l3 -0.541665   1.918096  -0.282  0.77791    
## s.l4    -0.044488   0.046802  -0.951  0.34290    
## dur.l4  -0.073315   0.100434  -0.730  0.46620    
## dlip.l4  2.558021   1.930105   1.325  0.18648    
## s.l5     0.006982   0.031979   0.218  0.82737    
## dur.l5   0.154565   0.088744   1.742  0.08300 .  
## dlip.l5  3.509294   1.786606   1.964  0.05080 .  
## ---
## Signif. codes:  0 '***' 0.001 '**' 0.01 '*' 0.05 '.' 0.1 ' ' 1
## 
## 
## Residual standard error: 0.2352 on 214 degrees of freedom
## Multiple R-Squared: 0.5078,  Adjusted R-squared: 0.4733 
## F-statistic: 14.72 on 15 and 214 DF,  p-value: < 2.2e-16 
## 
## 
## Estimation results for equation dlip: 
## ===================================== 
## dlip = s.l1 + dur.l1 + dlip.l1 + s.l2 + dur.l2 + dlip.l2 + s.l3 + dur.l3 + dlip.l3 + s.l4 + dur.l4 + dlip.l4 + s.l5 + dur.l5 + dlip.l5 
## 
##           Estimate Std. Error t value Pr(>|t|)    
## s.l1     0.0008782  0.0016122   0.545   0.5865    
## dur.l1  -0.0056531  0.0047009  -1.203   0.2305    
## dlip.l1  0.5565343  0.0903890   6.157 3.61e-09 ***
## s.l2     0.0015771  0.0023953   0.658   0.5110    
## dur.l2   0.0048147  0.0049965   0.964   0.3363    
## dlip.l2 -0.0693264  0.0984300  -0.704   0.4820    
## s.l3    -0.0021046  0.0024100  -0.873   0.3835    
## dur.l3   0.0041926  0.0050224   0.835   0.4048    
## dlip.l3  0.1924347  0.0969624   1.985   0.0485 *  
## s.l4     0.0006620  0.0023659   0.280   0.7799    
## dur.l4   0.0020471  0.0050771   0.403   0.6872    
## dlip.l4 -0.0596774  0.0975694  -0.612   0.5414    
## s.l5     0.0008779  0.0016166   0.543   0.5877    
## dur.l5  -0.0022074  0.0044862  -0.492   0.6232    
## dlip.l5 -0.0358022  0.0903153  -0.396   0.6922    
## ---
## Signif. codes:  0 '***' 0.001 '**' 0.01 '*' 0.05 '.' 0.1 ' ' 1
## 
## 
## Residual standard error: 0.01189 on 214 degrees of freedom
## Multiple R-Squared: 0.5119,  Adjusted R-squared: 0.4777 
## F-statistic: 14.96 on 15 and 214 DF,  p-value: < 2.2e-16 
## 
## 
## 
## Covariance matrix of residuals:
##             s       dur       dlip
## s     0.25167  0.026099 -0.0011302
## dur   0.02610  0.054949 -0.0019559
## dlip -0.00113 -0.001956  0.0001414
## 
## Correlation matrix of residuals:
##            s     dur    dlip
## s     1.0000  0.2219 -0.1895
## dur   0.2219  1.0000 -0.7017
## dlip -0.1895 -0.7017  1.0000
\end{verbatim}

We do not think a lag length of 5 is appropriate as the AIC suggests a
lag length of 1 is sufficient. Next, we check for serial correlation:

\begin{verbatim}
## 
##  Portmanteau Test (adjusted)
## 
## data:  Residuals of VAR object var_e
## Chi-squared = 70.315, df = 27, p-value = 1.011e-05
\end{verbatim}

We find that there is serial correlation in the specified model using
the adjusted portmanteau test.

\subsubsection{10 f:}\label{f}

Obtain the impulse response function from the model using
\(\Delta lip_t, \Delta ur_t\) and \(s_t\). Show that a positive shock to
the industrial production induces a decline in the unemployment rate
that lasts six quarters. Then, \(\Delta ur_t\) overshoots its long run
level before returning to zero.

\includegraphics{Assignment_3_files/figure-latex/unnamed-chunk-12-1.pdf}
\includegraphics{Assignment_3_files/figure-latex/unnamed-chunk-12-2.pdf}
\includegraphics{Assignment_3_files/figure-latex/unnamed-chunk-12-3.pdf}

\subsubsection{10 g:}\label{g}

Reverse the ordering and explain why the results depend on whether or
not \(\Delta lip_t\) proceeds \(\Delta ur_t\)

\includegraphics{Assignment_3_files/figure-latex/unnamed-chunk-14-1.pdf}
\includegraphics{Assignment_3_files/figure-latex/unnamed-chunk-14-2.pdf}
\includegraphics{Assignment_3_files/figure-latex/unnamed-chunk-14-3.pdf}

If \(\Delta lip_t\) proceeds \(\Delta ur_t\), then a contemporary effect
on \(\Delta lip_t\) affects \(\Delta ur_t\), but not vice versa. On the
other hand, if the reverse holds, then shocks to \(\Delta lip_t\) will
be delayed until next period before any noticeable change occurs to
\(\Delta ur_t\).


\end{document}
